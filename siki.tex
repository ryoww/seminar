\documentclass{ltjsarticle}
\usepackage[top=25truemm,bottom=25truemm,left=25truemm,right=25truemm]{geometry}
\usepackage[dvipdfmx]{graphicx}
\usepackage{listings, jvlisting}
\usepackage{subcaption}
\usepackage{siunitx}
\usepackage{float}
% \usepackage{amsmath}
% \usepackage{amssymb}
\usepackage[colorlinks=true, allcolors=blue]{hyperref}
\usepackage{mathtools,amsmath,amsthm,amssymb}
\usepackage{comment}
\usepackage{comment}
\usepackage{ulem}
\usepackage{cite} %cite拡張
\usepackage{latexsym} %記号追加
\usepackage{enumerate}	%列挙用
\usepackage{setspace} %余白制御
\usepackage{multirow}
\usepackage{luatexja-preset}


\lstset{
  basicstyle={\ttfamily},
  identifierstyle={\small},
  commentstyle={\smallitshape},
  keywordstyle={\small\bfseries},
  ndkeywordstyle={\small},
  stringstyle={\small\ttfamily},
  frame={tb},
  breaklines=true,
  columns=[l]{fullflexible},
  numbers=left,
  xrightmargin=8zw,
  xleftmargin=8zw,
  numberstyle={\scriptsize},
  stepnumber=1,
  numbersep=1zw,
  lineskip=-0.5ex
}


\begin{document}
\begin{align*}
  \frac{P_r}{P_t} = \left(\frac{\lambda}{4 \pi d}\right)^2 G_t G_r
\end{align*}

\begin{align*}
  \frac{P_r}{P_t} = \left(\frac{\lambda}{4 \pi d}\right)^2 = \frac{1}{\left(\frac{4 \pi d}{\lambda}\right)^2}
\end{align*}

\begin{align*}
  L_{fs} = 10 \log_{10} \left(\frac{4 \pi d}{\lambda}\right)^2 &= 20 \log_{10} \frac{4 \pi d}{\lambda} \\
  &= 20 \log_{10} \frac{4 \pi d f}{c}\\
  &= 20 \log_{10} \frac{4 \pi}{c} + 20 \log_{10} d + 20 \log_{10} f\\
  &= 20 \log_{10} d + 20 \log_{10} f - 148 [\mathrm{dB}]\\
  \because c = 3.0 \times 10^8
\end{align*}

\begin{tabular}{|c|c|c|c|}\hline
      & 住宅環境 & オフィス環境 & 商業施設環境 \\ \hline
  縦 $\times$ 横  & 10m $\times$ & 25m $\times$ 25m & 50m $\times$ 50m \\ \hline
  高さ & 2.5m & 3m & 3m \\ \hline
  床 & 木造,コンクリート & コンクリート & コンクリート \\ \hline
  壁面 & 木造,耐火ボード & 金属,コンクリート & コンクリート \\ \hline
\end{tabular}

\begin{align*}
  S_1
\end{align*}

% \newcolumntype{Y}{&gt;{\centering\arraybackslash}X}
% \begin{tabularx}{50mm}{|Y|Y|Y|Y|Y|Y|Y|}\hline
%   環境 & \multicolumn{2}{|c|}{集合住宅内} & \multicolumn{2}{|c|}{戸建て住宅内} & \multicolumn{2}{|c|}{オフィス内} \\ \hline
%   周波数[GHz] & 2.45 & 5.2 & 2.45 & 5.2 & 2.45 & 5.2 \\ \hline
%   N & 28 & 30 & 28 & 28 & 30 & 31 \\ \hline
%   $L_f[\mathrm{dB}]$ & $10^{*1}$ & $13^{*1}$ & $5$ & $7^{*2}$ & $14$ & $16$ \\ \hline
%   備考 &
%   \begin{tabular}{c}
%     *1:コンクリート壁\\1枚あたり
%   \end{tabular} &

%   \begin{tabular}{c}
%     *1:コンクリート壁\\1枚あたり
%   \end{tabular} &

%   & *2:木造モルタル
%   & & \\ \hline
% \end{tabularx}

\begin{tabular*}{150mm}{|c|c|c|c|c|c|c|}\hline
  環境 & \multicolumn{2}{|c|}{集合住宅内} & \multicolumn{2}{|c|}{戸建て住宅内} & \multicolumn{2}{|c|}{オフィス内} \\ \hline
  周波数[GHz] & 2.45 & 5.2 & 2.45 & 5.2 & 2.45 & 5.2 \\ \hline
  N & 28 & 30 & 28 & 28 & 30 & 31 \\ \hline
  $L_f[\mathrm{dB}]$ & $10^{*1}$ & $13^{*1}$ & $5$ & $7^{*2}$ & $14$ & $16$ \\ \hline
  備考 &
  \begin{tabular}{c}
    *1:コンクリート壁\\1枚あたり
  \end{tabular} &

  \begin{tabular}{c}
    *1:コンクリート壁\\1枚あたり
  \end{tabular} &

  & *2:木造モルタル
  & & \\ \hline
\end{tabular*}

\begin{tabular}{|c|c|c|c|c|c|c|}\hline
  環境 & \multicolumn{2}{|c|}{集合住宅内} & \multicolumn{2}{|c|}{戸建て住宅内} & \multicolumn{2}{|c|}{オフィス内} \\ \hline
  周波数[GHz] & 2.45 & 5.2 & 2.45 & 5.2 & 2.45 & 5.2 \\ \hline
  N & 28 & 30 & 28 & 28 & 30 & 31 \\ \hline
  $L_f[\mathrm{dB}]$ & $10^{*1}$ & $13^{*1}$ & $5$ & $7^{*2}$ & $14$ & $16$ \\ \hline
  備考 &
  \begin{tabular}{c}
    *1:コンクリート壁\\1枚あたり
  \end{tabular} &

  \begin{tabular}{c}
    *1:コンクリート壁\\1枚あたり
  \end{tabular} &

  & *2:木造モルタル
  & & \\ \hline
\end{tabular}

\begin{align*}
  S_1 &= \frac{h_{22}}{h_{11}h_{22} - h_{12}h_{21}} \cdot y_1 - \frac{h_{12}}{h_{11}h_{22} - h_{12}h_{21}} \cdot y_2\\
  S_2 &= - \frac{h_{21}}{h_{11}h_{22} - h_{12}h_{21}} \cdot y_1 + \frac{h_{11}}{h_{11}h_{22} - h_{12}h_{21}} \cdot y_2
\end{align*}

% \begin{equation*}
%   \begin{bmatrix*}
%     y_1 \\ y_2
%   \end{bmatrix*}

%   \begin{bmatrix*}
%     h_{11} & h_{12} \\
%     h_{21} & h_{22}
%   \end{bmatrix*}

% \end{equation*}

\begin{equation*}
  \begin{bmatrix}
    y_1 \\
    y_2
    \end{bmatrix}
    =
    \begin{bmatrix}
    h_{11} & h_{12} \\
    h_{21} & h_{22}
    \end{bmatrix}
    \cdot
    \begin{bmatrix}
    s_1 \\
    s_2
    \end{bmatrix}
    +
    \begin{bmatrix}
      n_1\\
      n_2
    \end{bmatrix}
\end{equation*}

\begin{equation*}
  \begin{bmatrix}
    s_1 \\
    s_2
    \end{bmatrix}
    =
    \begin{bmatrix}
    h_{11} & h_{12} \\
    h_{21} & h_{22}
    \end{bmatrix}^{-1}
    \cdot
    \begin{bmatrix}
    y_1 \\
    y_2
    \end{bmatrix}
\end{equation*}

\begin{align*}
  G &= HH^H\\
    % &=
    % \begin{bmatrix}
    %   |h_{11}|^2+|h_{12}|^2 & h_{11}h_{21}^*+h_{12}h_{22}^*\\
    %   h_{11}^*h_{21}+h_{12}^*h_{22} & |h_{21}|^2+|h_{22}|^2
    % \end{bmatrix}
\end{align*}

\begin{align*}
  H &= UDV^H\\
    &=
    \begin{bmatrix}
      u_1 & u_2
    \end{bmatrix}
    \begin{bmatrix}
      \sqrt{\lambda_1} & 0\\
      0 & \sqrt{\lambda_2}
    \end{bmatrix}
    \begin{bmatrix}
      v_1 & v_2
    \end{bmatrix}^H
\end{align*}

\section*{株式会社 ゼネット}
\begin{itemize}
  \item 設立\\1999年
  \item 技術者合計\\198名
  \item 事業内容\\金融機関のシステム開発や保守,ERP事業,\\AWS事業や研修事業まで幅広く活躍.
\end{itemize}
\end{document}